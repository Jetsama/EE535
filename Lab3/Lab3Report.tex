\documentclass[10pt,a4paper]{article}
\usepackage[utf8]{inputenc}
\usepackage{amsmath}
\usepackage{amsfonts}
\usepackage{amssymb}
\usepackage{tikz}
\usetikzlibrary{datavisualization}
\usetikzlibrary{datavisualization.formats.functions}
\usepackage[left=25mm,right=25mm]{geometry}
\usepackage{pgfplots}
\usepgfplotslibrary{external} 
\tikzexternalize
\usepackage{float} 
\usepackage{multicol}
\usepackage{hyperref}


\graphicspath{ {./figures/} }

\begin{document}
\begin{multicols}{2}
\newenvironment{indentPar}[1]%
 {\begin{list}{}%
         {\setlength{\leftmargin}{#1}}%
         \item[]%
 }
 {\end{list}}

\begin{flushleft}
\begin{LARGE}EE 535 Lab 3: Hall Effect Mobility
\end{LARGE}	
\\Jonathan Hess
\\\href{https://github.com/Jetsama/EE535/tree/main/Lab3}{https://github.com/Jetsama/EE535/tree/main/Lab3}
\end{flushleft}


\section*{Abstract}
The hall effect
The precise thickness of a thin absorbing wafer can be found using the transmission observed from a spectrophotometer. This procedure was done for amorphous silicon (aSi) and zinc selenide (ZnSe) wafers with unknown widths.





\section*{Introduction}

Two wafers of different materials were observed with a spectrophotometer to yield measurements of The method evaluated in this paper to calculate the wafer film's width uses the transmition measured as well as the peaks and valleys of the resulting graph. 




\section*{Definitions}
Quantum Effi
Transmission (T) is the amount of light and electromagnetic radiation that passes through a media.\\
Reflection (R) is the amount of light and electromagnetic radiation that changes direction. \\
Absorbance (A) is the measure of how much light is absorbed by a substance at a particular wavelength.\\



\section*{Experimental}
During this lab measurements of 





\section*{Theory}
The hall effect is created by the Lorenz force. Electrons moving along an electric field perpendicular to a magnetic field. The force of the magnetic field applies is calculated using equation \ref{eq:lorentz}. Where q is the elemental charge of an electron, E is the electric field, and B is the magnetic field.


\begin{equation}\label{eq:lorentz}
F = qE + qv*B
\end{equation}


Majority carrier type of device under test

carrier concentration
carrier mobility








\section*{Results}


\section*{Appendix}
For information on the pure data or computational scripts there is a repository for this and other labs. \\
https://github.com/Jetsama/EE535\\
\begin{thebibliography}{9}
\bibitem{book}
 Solid State Electronic Devices (2006, Prentice Hall), Ben Streetman, Sanjay Banerjee
\bibitem{Lasers}
An Entry Level Guide to Understanding Lasers (2008), Chapter 9.2.4, Jeff Hecht, 3rd ed. 

\bibitem{paper}
R Swanepoel 1983 J. Phys. E: Sci. Instrum. 16 1214

\bibitem{optical}
Optical Processes in Semiconductors, Jacques I. Pankove

\bibitem{carry}
Cary 100/300/4000/5000/6000i/7000 Spectrophotometers User's Guide

\bibitem{tauc}
How To Correctly Determine the Band Gap Energy of Modified Semiconductor Photocatalysts Based on UV–Vis Spectra, Patrycja Makuła, Michał Pacia, and Wojciech Macyk\\https://pubs.acs.org/doi/10.1021/acs.jpclett.8b02892
\bibitem{semi}
Silicon Photo Multipliers Detectors Operating in Geiger Regime: an Unlimited Device for Future Applications,Giancarlo Barbarino, Riccardo de Asmundis, Gianfranca De Rosa, Carlos Maximiliano Mollo, Stefano Russo and Daniele Vivolo


\bibitem{peaks} \href{https://erdogant.github.io/findpeaks/pages/html/Examples.html#find-peaks-in-low-sampled-dataset}{text}


\bibitem{marq} E Marquez et al 1992 J. Phys. D: Appl. Phys. 25 535
\bibitem{siK}Handbook of Optical Constants of Solids, Edward D. Palik. Academic Press, Boston, 1985 
\end{thebibliography}


\end{multicols}
\end{document}
